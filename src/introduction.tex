% Introduction.tex
% !TEX root = ../main.tex
\newpage
\begin{center}
    {\large\textbf{LỜI MỞ ĐẦU}}
\end{center}
\onehalfspacing



% Giới thiệu bối cảnh và tầm quan trọng của đề tài
Trong bối cảnh nhu cầu di chuyển liên tỉnh tại Việt Nam ngày càng gia tăng, đặc biệt trong các dịp lễ, Tết hoặc mùa du lịch, việc đặt vé xe khách thông qua các nền tảng trực tuyến đã trở thành xu hướng tất yếu, mang lại sự tiện lợi, nhanh chóng và minh bạch cho người dùng. Tuy nhiên, thực tế cho thấy nhiều hệ thống đặt vé trực tuyến hiện nay vẫn còn tồn tại các hạn chế như giao diện phức tạp, quy trình đặt vé rườm rà, hoặc thiếu sự tương thích với thói quen và kỳ vọng của người dùng Việt Nam. Những vấn đề này không chỉ gây khó khăn trong quá trình sử dụng mà còn làm giảm độ tin cậy và mức độ phổ biến của các dịch vụ trực tuyến.

% Trình bày mục tiêu và ý nghĩa của đề tài
Xuất phát từ thực trạng trên, đồ án ``Thiết kế giao diện hệ thống đặt vé xe khách'' được thực hiện với mục tiêu xây dựng một giao diện web thân thiện, trực quan và tối ưu hóa trải nghiệm người dùng (UX). Giao diện này cho phép người dùng dễ dàng thực hiện các thao tác từ việc lựa chọn điểm đi, điểm đến, thời gian khởi hành, hãng xe, loại ghế, đến hoàn tất thanh toán một cách mượt mà và hiệu quả. Đồ án tập trung áp dụng các nguyên lý thiết kế UX được đề cập trong tài liệu \textit{The Design of Everyday Things} của Don Norman, bao gồm các khái niệm cốt lõi như khả năng gợi ý (\textit{affordance}), khả năng hiển thị (\textit{visibility}), phản hồi tức thời (\textit{feedback}), và sự tương thích với mô hình tinh thần của người dùng (\textit{mental model}). Những nguyên lý này được vận dụng một cách có hệ thống nhằm đảm bảo giao diện không chỉ dễ sử dụng mà còn mang lại cảm giác thân thiện, tự nhiên và phù hợp với văn hóa, thói quen của người dùng Việt Nam.

% Mô tả phương pháp thực hiện
Để đạt được mục tiêu đề ra, đồ án tuân thủ quy trình phát triển phần mềm chuyên nghiệp, kết hợp giữa nghiên cứu lý thuyết và ứng dụng thực tiễn. Nhóm đã tiến hành khảo sát thực tế, phân tích hành vi và nhu cầu của người dùng, đồng thời tham khảo các hệ thống đặt vé hiện có để xác định các điểm mạnh và hạn chế. Quy trình thiết kế được thực hiện theo mô hình lặp \textit{design-test-iterate}, bao gồm các giai đoạn thiết kế, thử nghiệm và cải tiến liên tục nhằm đảm bảo sản phẩm đáp ứng tốt các yêu cầu kỹ thuật và trải nghiệm người dùng. Ngoài ra, đồ án cũng chú trọng tích hợp các công nghệ hiện đại như thiết kế giao diện thích ứng (\textit{responsive design}) để đảm bảo tương thích trên nhiều thiết bị, tối ưu hóa tốc độ tải trang, và đảm bảo tính bảo mật trong các giao dịch trực tuyến.

% Nhấn mạnh ý nghĩa và đóng góp của đồ án
Thông qua việc thực hiện đồ án, nhóm không chỉ hướng đến việc tạo ra một sản phẩm hoàn thiện, đáp ứng nhu cầu thực tiễn của thị trường, mà còn nhằm nâng cao kiến thức và kỹ năng của các thành viên trong các lĩnh vực thiết kế giao diện người dùng (UI), trải nghiệm người dùng (UX), lập trình web và quản lý dự án. Đồ án cũng là cơ hội để các thành viên rèn luyện tư duy phân tích, kỹ năng giải quyết vấn đề và khả năng làm việc nhóm trong môi trường thực tế, từ đó chuẩn bị tốt hơn cho các thách thức trong ngành công nghệ thông tin và thiết kế.

% Lời cảm ơn
Nhóm xin gửi lời cảm ơn chân thành đến giảng viên Nguyễn Thành Luân tại khoa Khoa học và kỹ thuật thông tin đã tận tình hướng dẫn và hỗ trợ, người đã cung cấp những ý kiến quý báu và định hướng rõ ràng, giúp nhóm hoàn thiện sản phẩm một cách tốt nhất trong suốt quá trình thực hiện đồ án.
